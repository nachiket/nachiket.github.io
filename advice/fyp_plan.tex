\section{NTU SCE Final Year Project - Plan}

\begin{figure}[htbp]
\centering
\includegraphics{ntu.jpg}
\end{figure}

\textbf{Student}: Donald Duck \textbf{Supervisor}:
Dr.~Nachiket Kapre \textbf{Date}: 11th August 2013

Download Markdown template fyp\_plan.md.html
\emph{Sharepoint forces .html extension. Rename .html to .md
on download}

\begin{quote}
wget
http://www.ntu.edu.sg/home/nachiket/advice/fyp\_plan.md.html
\end{quote}

\begin{quote}
mv fyp\_plan.md.html fyp\_plan.md
\end{quote}

To compile on Mac OS X or Ubuntu Linux with
\href{http://johnmacfarlane.net/pandoc}{Pandoc} installed,
use

\begin{quote}
pandoc fyp\_plan.md -o fyp\_plan.pdf
\end{quote}

\textbf{Grading Policy}

\begin{longtable}[c]{@{}llp{4cm}@{}}
\hline\noalign{\medskip}
10\% & Regular 1:1 meetings & The process of conducting
yourself through the FYP is as important as the final
product. You will be expected to attend all 1:1 weekly
meetings. Exceptions only with MCs.
\\\noalign{\medskip}
10\% & Draft plan & A good engineer knows how to provision
the most important resource: \textbf{time}, for his project.
I will expect at least a rough plan outline that we revise
as we execute the project.
\\\noalign{\medskip}
20\% & Regular 15-minute group presentations (once a month)
& A good engineer is also a good oral communicator. I will
expect short 15-minute presentations to my research group
once a month. Ultimately, I want you to become confident
speakers and great communicators..
\\\noalign{\medskip}
30\% & Final report & Technical report writing is very
different from writing an exam. In industry, you will be
expected to be precise with your specifications and
communications. At the same time, you need to tell a
\textbf{good story}. You need to connect with your reader at
several levels.
\\\noalign{\medskip}
20\% & Final presentation & This should be the
ultimate/improved version of your regular group
presentations
\\\noalign{\medskip}
\hline
\end{longtable}

\section{Abstract (200 words)}

\begin{itemize}
\itemsep1pt\parskip0pt\parsep0pt
\item
  Crisp, quantitative Claim of what you are planning to show

  \begin{itemize}
  \itemsep1pt\parskip0pt\parsep0pt
  \item
    e.g.~We can make SPICE run 10 times faster using cheap
    off-the-shelf hardware (\textbf{correct})
  \item
    e.g.~SPICE runs faster when using our ideas
    (\textbf{wrong})
  \end{itemize}
\end{itemize}

\section{Perspective (1/2 page)}

\begin{itemize}
\itemsep1pt\parskip0pt\parsep0pt
\item
  What do you want to learn from this project?
\item
  How do you expect your advisor to help you out?
\item
  What is the one thing you truly love about computer
  engineering? Feel free to describe a topic outside the
  scope of the FYP.
\end{itemize}

\section{Planning (2 pages)}

\begin{itemize}
\itemsep1pt\parskip0pt\parsep0pt
\item
  Include your term calendar. The advisor should have an
  idea of how loaded you are during the semester and how
  much effort/work can you put in. It is ok to attach just a
  JPG or screenshot of your StudentLink calendar.
\item
  Also attach a rough sketch of your proposed plan. Try to
  use a simple Table format organized by month.
\item
  When you are actually done, attach another plan of actual
  execution.

  \begin{itemize}
  \itemsep1pt\parskip0pt\parsep0pt
  \item
    This should hopefully teach you to plan your projects
    better for the future.
  \end{itemize}
\item
  Feel free to fill the following template directly in
  Markdown.
\end{itemize}

\begin{longtable}[c]{@{}ll@{}}
\hline\noalign{\medskip}
Month 1 & Foraging for berries and fruit
\\\noalign{\medskip}
Month 2 & Building hut for shelter against rain\ldots{}
\\\noalign{\medskip}
Month 12 & Get rescued from island
\\\noalign{\medskip}
\hline
\end{longtable}

\section{Resources (1/2 page)}

\begin{itemize}
\itemsep1pt\parskip0pt\parsep0pt
\item
  What hardware/software infrastructure do you need for your
  project?
\item
  Are you comfortable with these platforms? How will you
  learn about these new platforms?
\end{itemize}

\section{Reading List (1 pages)}

\begin{itemize}
\itemsep1pt\parskip0pt\parsep0pt
\item
  Talk to your advisor to get a reading list. By the end of
  the project, each of the suggested references should
  become short paragraphs of description.
\end{itemize}

 Updated: 11th August 2013 
